\section{Trénovací API}
\label{sec:Chapter52}
Pro trénink, export, inferenci a všechny ostatní manipulace s modely byla použita knihovna TensorFlow 2.13.0. Na základě této knihovny jsem vyvinul svou kódovou bázi pro účely strojového učení této práce fungující primárně na příkazové řádce pro trénink, evaluaci, generaci metrik a grafů, ale i interaktivní inferenci. Tato kódová báze sloužila jako pracovní nástroj pro práci se sítěmi aplikovanými v této práci a s pomocnými funkcemi pro generování výsledků, evaluací a grafů. Společně s knihovnou TensorFlow byly využity přídavné moduly jako je např. \textit{tensorflow-addons}, \textit{tensorflow-probability} \cite{tensorflow_libs} pro využití při programování částí tréninku či inference či \textit{TensorRT} \cite{tensorrt_docs} sloužící pro zlepšení inferenčního výkonu a rychlosti.
\endinput