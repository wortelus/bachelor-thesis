\section{Parametry trénování}
\label{sec:Chapter54}
Tréninky a evaluace modelů konvolučních neuronových sítí v této práci využívají připraveného datasetu, který byl rozdělen na 3 části:
\begin{enumerate}
    \item \textbf{Trénovací část}, 60 \% celkového datasetu, sloužící pro aktualizaci a trénování parametrů sítě
    \item \textbf{Validační část}, 20 \%, sloužící pro evaluaci epoch během trénování konvolučních neuronových sítí a včasnému načasování zastavení tréninku
    \item \textbf{Testovací část}, 20 \%, sloužící pro finální generaci metrik a porovnání modelů. Jedná se právě o tyto data, která síť během tréninku ``neviděla'' ani jednou, takže riziko vzniku jakéhokoli tzv. biasu vůči datům během finální evaluace či porovnání je minimální
\end{enumerate}

Data, které byly reprezentovány v CSV souborech pro každou třídu separátně (blíž popsány v kapitole \ref{sec:Chapter31}) byly celkově načteny do jednoho objektu typu \texttt{tf.TextLineDataset}. Po načtení byly nejdříve náhodně zamíchány, ověřeny zda neobsahují třídní nevyvážení (ang. class imbalance) pomocí porovnání poměrů všech 11 tříd mezi sebou, zkontrolovány na úniky dat a uloženy do 3 odpovídajících CSV souborů, které pak byly použity opakovaně pro všechny modely.

Pro trénink všech našich sítí byl použit optimizér Adam s mírou učení (ang. learning rate) nastavenou na $1.0e^{-3}$. Z důvodů velikosti VRAM na domácím tréninkovém stroji a jako rozumný odhad jsem nastavil velikost dávky\footnote{Množství dat, která jsou zpracována a propagována sítí před aktualizací trénovatelných parametrů.} (ang. batch size) při tréninku na adekvátní hodnotu 32.

\endinput