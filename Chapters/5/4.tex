\section{Průběh tréninku}
\label{sec:Chapter54}
Tréninky byly spouštěny ve stavech specifikovaných v kapitole \ref{sec:Chapter53} na stroji popsaném v kapitole \ref{sec:Chapter51}. Pro trénink byly v prostředí \texttt{Keras} použity zpětné volání a výstupy pro:
\begin{itemize}
    \item Včasné zastavení pro případ nezlepšení výsledků validační ztráty mezi epochami, nastaveno na 4 epochy.
    \item Ukládání snímku trénovatelných vah a gradientů z dosavadně nejlepší epochy dle validační ztráty - \texttt{best.ckpt}
    \item Ukládání snímku trénovatelných vah a gradientů z poslední epochy dle validační ztráty - \texttt{best.ckpt}
    \item Zaznamenování trénovacích a validačních ztrátových hodnot
    \item Trénovací a validační výstup pro nástroj \texttt{TensorBoard}
\end{itemize}
Během tréninku jsme díky použití několika verzí architektur sítí s úpravami ve ztrátových funkcích zastavovali model i manuálně při signifikantním nezlepšení ztrátové hodnoty. Ve všech případech byl pro výsledný model zvolen tzv. checkpoiint s nejlepší validační ztrátou.
\endinput