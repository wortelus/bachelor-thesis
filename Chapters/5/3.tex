\section{Parametry trénování}
\label{sec:Chapter53}
Tréninky a evaluace modelů konvolučních neuronových sítí v této práci využívají připraveného datasetu, který byl rozdělen na 3 části:
\begin{enumerate}
    \item \textbf{Trénovací část}, 60 \% celkového datasetu, sloužící pro aktualizaci a trénování parametrů sítě.
    \item \textbf{Validační část}, 20 \%, sloužící pro evaluaci epoch během trénování konvolučních neuronových sítí a včasnému načasování zastavení tréninku.
    \item \textbf{Testovací část}, 20 \%, sloužící pro finální generování metrik a porovnání modelů. Jedná se právě o tyto data, která síť během tréninku \enquote{neviděla} ani jednou, takže riziko vzniku jakéhokoli tzv. biasu vůči datům během finální evaluace či porovnání je minimální.
\end{enumerate}

Data v CSV souborech (blíže popsána v kapitole \ref{sec:Chapter31}) byla celkově načtena do jednoho objektu typu \texttt{tf.TextLineDataset}. Data byla nejdříve zamíchána, ověřena na třídní vyváženost, zkontrolována na úniky dat a nakonec uložena do tří CSV souborů pro opakované použití ve všech modelech.

Pro trénink všech našich sítí byl použit optimizér Adam s mírou učení (ang. learning rate) nastavenou na 0,001. Z důvodů velikosti VRAM na domácím tréninkovém stroji a jako rozumný odhad jsem nastavil velikost dávky\footnote{Množství dat, která jsou zpracována a propagována sítí před aktualizací trénovatelných parametrů.} (ang. batch size) při tréninku na adekvátní hodnotu 32.

\endinput