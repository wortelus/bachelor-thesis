\section{Jazyky a prostředí}
\label{sec:Chapter51}
V rámci této práce jsem se pohyboval primárně v prostředí jazyka \textbf{Python}, což je skvělá volba pro rychlé, spolehlivé, nastavitelné a silně modulární nastavení v oblasti strojového učení. S jazykem Python jsme používali nástroje virtuálních prostředí \textit{pip} a \textit{Anaconda}, které spravovaly balíčky jazyka Python pro tuto práci

Pro vývoj byla použita verze Python 3.10 na operačním systému Ubuntu 22.04.03 LTS\footnote{Dlouhodová podpora, ang. Long Time Support.}. Operační systém Ubuntu nebyl první volbou. První volbou bylo prostředí Debian 12. Avšak ukázalo se, že budování prostředí co se týče balíčkovacích kompatibilit mezi verzemi ovladačů, verzemi balíčků pro Python a verzemi cuDNN\footnote{Knihovna zprostředkující jazyk CUDA grafických karet NVIDIA pro účely strojového učení.} bylo co se týče instalace a nastavení nejvíce vyhovující na prostředí Ubuntu, což bylo doporučeno i dokumentací následně zmíněných knihoven (jako je např. TensorFlow).

Toto prostředí nebylo virtualizováno, nebylo využito prostředků jako je např. Docker a tento systém byl uzpůsoben přímo pro využití v analýze dat a strojového učení.

Většina experimentálního trénování proběhla na následujícím hardwaru:

\begin{table}[ht]
\centering
\begin{tabular}{@{}lll@{}}
\toprule
Součást & Model \\
\midrule
CPU & AMD Ryzen 5 7600 6-core processor \\
GPU & NVIDIA GTX 1070 Strix 8GB VRAM  \\
RAM & 32 GB 6000MHz DDR5 \\
\bottomrule
\end{tabular}
\caption{Domácí systém pro trénink}
\label{fig:wortelus_pc}
\end{table}
\endinput