\section{Jazyky a prostředí}
\label{sec:Chapter51}
V rámci této práce jsem se pohyboval primárně v prostředí jazyka \textbf{Python}, což je skvělá a častá volba pro mnoho úloh analýzy dat a strojového učení. S jazykem Python bylo použito nástroje virtuálních prostředí \textit{pip} a \textit{Anaconda}, které spravovaly balíčky jazyka Python pro tuto práci.

Pro vývoj byla použita verze Python 3.10 na operačním systému Ubuntu 22.04.03 LTS\footnote{Dlouhodobá podpora, ang. Long Time Support.}. Budování prostředí, co se týče balíčkovacích kompatibilit mezi verzemi ovladačů, verzemi balíčků pro Python a verzemi cuDNN\footnote{Knihovna zprostředkující jazyk CUDA grafických karet NVIDIA pro účely strojového učení.}, bylo co se týče instalace a nastavení nejvíce vyhovující prostředí Ubuntu, což bylo doporučeno i dokumentací následně zmíněných knihoven (jako je např. TensorFlow) jako volba linuxové distribuce \cite{tensorflow_install}.

Toto prostředí nebylo virtualizováno, nebylo využito prostředků jako je např. Docker a tento systém byl uzpůsoben přímo pro využití v analýze dat a strojového učení.

Všechny experimentální trénování proběhly na následujícím hardwaru:

\begin{table}[ht]
\centering
\begin{tabular}{@{}lll@{}}
\toprule
Součást & Model \\
\midrule
CPU & AMD Ryzen 5 7600 6-core processor \\
GPU & NVIDIA GTX 1070 Strix 8 GB VRAM  \\
RAM & 32 GB 6000 MHz DDR5 \\
\bottomrule
\end{tabular}
\caption{Použitý hardware pro trénink}
\label{fig:wortelus_pc}
\end{table}

\subsection{Dodatečné nástroje}
\label{sec:Chapter511}
Bylo také použito několika nástrojů komplementárním k postupu na úlohách řešených v rámci této práce, mezi ně patří:
\begin{itemize}
    \item \textbf{NumPy} -- jedna z nejpoužívanějších knihoven pro vědecké a statistické výpočty v jazyce Python. Obsahuje různé operace pro práci s poli, vektory a více-dimenzionálními strukturami (tenzory).
    \item \textbf{Seaborn} -- knihovna pro jazyk Python pro rychlé generování diagramů, grafů a podobných grafických vizualizací. Je navržena tak, aby byla snadno použitelná a zároveň poskytovala dynamické informativní výstupy.
    \item \textbf{Matplotlib} -- vykreslovací Python knihovna pro detailnější a pokročilejší práci pro generování diagramů, grafů, vizuálních prvků a mnoho dalšího s vysokým stupněm přizpůsobení poskytující možnost hlubší analýzy a prezentace dat.
    \item \textbf{Pandas} -- knihovna jazyka Python pro práci se strukturovanými textovými daty pro využití v analýze dat a strojovém učení.
    \item \textbf{TensorBoard} -- vizualizační nástroj integrovaný s nástrojem TensorFlow. Umožňuje sledovat a vizualizovat různé metriky jako jsou gradienty, váhy neuronových sítí a další data dostupné z trénovacího procesu.
\end{itemize}
\endinput