\chapter{Závěr}
\label{sec:Chapter7}
Hlavním cílem této práce byla rešerše přístupů k lokalizaci klíčových bodů pomocí hlubokých neuronových sítí a následný pokus implementace vybraných modelů. Modely byly úspěšně implementovány pomocí knihovny a učícího API TensorFlow.

Práce byla zaměřena primárně na architektury hlubokých konvolučních neuronových sítí U-Net. Implementována byla síť U-Net a U-Net++. V rámci představení byl i vytvořen v rámci této práce model U-Net STN pro adresování spaciálních invariancí a vyzkoušení jeho výkonu na testovacích syntetických snímcích a reálných snímcích.

Model U-Net dosáhl obstojných výsledků a představení modulu STN do sítě U-Net ukázalo zajímavé zlepšení zejména v přesnosti lokalizace. Obstojila si i síť s hlubší architekturou U-Net++ bez použití hluboké supervize, která dosáhla nejnižšího počtu falešných detekcí a průměru chyby lokalizace.
\endinput