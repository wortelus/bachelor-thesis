\chapter{Závěr}
\label{sec:Chapter7}
Hlavním cílem této práce byla rešerše přístupů k lokalizaci klíčových bodů pomocí hlubokých neuronových sítí a následná implementace vybraných modelů. Modely byly úspěšně implementovány pomocí knihovny TensorFlow a jazyka Python 3.

Práce byla zaměřena primárně na architektury hlubokých konvolučních neuronových sítí U-Net. Implementována byla síť U-Net a U-Net++. V rámci této práce byl také vytvořen model U-Net STN a bylo provedeno praktické ověření jeho výkonu na testovacích syntetických a reálných snímcích.

Model U-Net dosáhl na syntetických snímcích konzistentních výsledků korektní lokalizace a klasifikace klíčových bodů. Představení modulu STN do sítě U-Net vykázalo zajímavé změny do přesnosti lokalizace. Nejnižší průměrné chyby lokalizace a nejmenšího počtu falešných detekcí dosáhla síť s hlubší architekturou U-Net++ bez použití hluboké supervize, která výsledně měla chybu lokalizace 0,69$\pm$1,20px a nulový počet falešných detekcí od hranice síly lokalizace 0,9. Dosáhla také lepších výsledků při zkouškách na reálně vyfocených snímcích. Síť DINOv2 na snímku \ref{fig:dinov2_sparse_car_lego} dosáhla korektní korespondence na většině detekovaných klíčových bodech. Všechny sítě měly variační koeficient lokalizace od 1,0 po 1,8 a průměry chyb lokalizace se většinou pohybovaly okolo 1 pixelu. Falešné detekce byly v řádech desítek na 110 000 snímků u všech modelů. 

Tato práce prozkoumala možnosti hlubokých sítí U-Net a prostorového transformeru STN, avšak existuje prostor pro zlepšení výkonu všech implementovaných sítí. Výsledky nepředstavují plný potenciál zkoumaných možností. Vylepšení v oblasti augmentace dat, ladění parametrů, architektury, změn do fáze tréninku atd. jsou schopny zlepšit schopnosti implementovaných hlubokých sítí. Sítě nepředstavily dostatečnou prostorovou invarianci na reálných snímcích, pravděpodobně ze zmíněných důvodů. Práce může sloužit jako podklad pro další studium a zlepšování architektur rodiny U-Net v kontextu lokalizace zájmových objektů, resp. jejich klíčových bodů, v obraze.
\endinput