\chapter{Závěr}
\label{sec:Chapter7}
Hlavním cílem této práce byla rešerše přístupů k lokalizaci klíčových bodů pomocí hlubokých neuronových sítí a následný pokus implementace vybraných modelů. Modely byly úspěšně implementovány pomocí knihovny a učícího API TensorFlow v jazyce Python 3.

Práce byla zaměřena primárně na architektury hlubokých konvolučních neuronových sítí U-Net. Implementována byla síť U-Net a U-Net++. Byl i vytvořen v rámci této práce model U-Net STN a vyzkoušení jeho výkonu na testovacích syntetických a reálných snímcích.

Model U-Net dosáhl obstojných výsledků a představení modulu STN do sítě U-Net ukázalo zajímavé zlepšení zejména v přesnosti lokalizace na testovacím datasetu. Nejlépe si obstojila síť s hlubší architekturou \textbf{U-Net++ bez použití hluboké supervize}, která dosáhla nejnižšího počtu falešných detekcí a průměru chyby lokalizace (0,69 $\pm$ 1.20). Dosáhla také lepších výsledků při zkouškách na reálně vyfocených snímcích. \textbf{Síť DINOv2 na snímku \ref{fig:dinov2_sparse_car_lego} dosáhla korektní korespondence na většině detekovaných klíčových bodech.}

Sítě U-Net, U-Net++ bez HS a sítě U-Net STN dosáhly uspokojivých výsledků na testovacím datasetu. Všechny sítě představovaly \textbf{variční koeficient detekce od 1,0 po 1,8} a průměry chyb detekce se většinou pohybovaly \textbf{okolo 1 pixelu}. \textbf{Falešné detekce byly v řadách desítek na 110 000 snímků.} 

Tato práce dostatečně prozkoumala možnosti hlubokých sítí U-Net a prostorového transformeru STN, avšak existuje prostor pro zlepšení výkonu všech implementovaných sítí. Výsledky nepředstavují plný potenciál zkoumaných možností. Vylepšení v oblasti augmentace dat, tuningu architektury či jemných změn do fáze tréninku atd. jsou schopny zlepšit schopnosti implementovaných hlubokých sítí. Sítě nepředstavily dostatečnou spaciální invarianci na reálných snímcích, pravděpodobně ze zmíněných důvodů.

Tato práce může sloužit jako podklad pro další studování či zlepšování spaciální invariance v hlubokém učení pro analýzu obrazu, přehled metod pro lokalizaci a korespondenci klíčových bodů nebo popis a průzkum sítí U-Net.
\endinput