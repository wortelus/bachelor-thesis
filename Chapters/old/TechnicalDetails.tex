\chapter{Technické detaily}
\section{Křížové odkazy}
\label{sec:CrossReferences}
Odborné texty, mezi které lze počítat i bakalářské, diplomové a disertační práce, obvykle obsahují množství křížových odkazů odkazující na nejrůznější části textu:
\begin{description}
	\item [kapitoly] -- například odkaz na kapitolu \ref{sec:Uherske}. Pokud odkazujeme na kapitolu, která je značně vzdálená od současné stránky, bývá dobrým zvykem k odkazu na číslo kapitoly přidat ještě i odpovídající číslo stránky, jako například pokud odkazujeme na kapitolu \ref{sec:Introduction} na straně \pageref{sec:Introduction}.
	
	\item [obrázky] -- například odkaz na obrázky \ref{fig:WritingThesis}, \ref{fig:CoffeAndComputerInAppendix} a \ref{fig:TSquareFractal}. Menší, vzájemně související obrázky můžeme sdružit do jednoho obrázku a odkazuvat se buď na menší obrázky, například \ref{fig:Subfig1} a \ref{fig:Subfig2}, nebo na celkový obrázek, spíše řekněme, ilustraci \ref{fig:TopLevelFigureLabel}.
	
	\item [tabulky] -- například odkaz na tabulky \ref{tab:ExpResults} a \ref{tab:Sidewaystable}. Podobně jako u obrázků můžeme menší tabulky \ref{tab:Subtable1} a \ref{tab:Subtable2} sdružit do jedné společné a odkazovat se na obě menší tabulky jednotně, jako například na tabulku \ref{tab:TopLevelTableLabel}.
	
	\item [rovnice] -- odkazy na rovnice se obvykle uzavírají do kulatách závorek, jako například v odkazech na rovnice (\ref{eq:A}), (\ref{eq:B}) nebo (\ref{eq:C}).
	
	\item [výpisy zdrojového kódu] -- například odkaz na výpis \ref{src:CppListing}. Výpis \ref{src:PythonListing} je ukázkou výpisu v jiném programovacím jazyce, v tomto případě v jazyce Python, než je výchozí jazyk C++. Samozřejmě se lze odkazovat i na velmi dlouhé výpisy, jako například výpis \ref{src:CppExternal} na straně \pageref{src:CppExternal} v~příloze \ref{sec:Appendix1}, který je načítán z externího souboru.
\end{description}

\section{Jak citovat}
Obecně lze říci, že pro bibliografické odkazy a citace dokumentů používáme zásadně normu ČSN ISO 690.
\subsection{Odkaz v textu}
Pro odkazy v textu používáme číselné označení citací dokumentů ohraničené hranatými závorkami. Takže například můžeme citovat časopisecké \emph{články} \cite{herrmann, bertram, moore, yoon, sigfridsson, baez/article}, \emph{knihy} \cite{wilde, nietzsche:ksa1, averroes/bland, hammond, cotton, knuth:ct:a, gerhardt, gonzalez, companion}, \emph{periodika} \cite{jcg}, \emph{bakalářské, diplomové či diserteční práce} \cite{geer}, \emph{patenty} \cite{kowalik, almendro, sorace, laufenberg}, \emph{online zdroje} \cite{ctan, wassenberg, itzhaki, markey, baez/online} či \emph{manuály} \cite{cms}.

\subsection{Seznam citací}
Seznam citací je umístěn na konci závěrečné práce, před přílohami, a musí obsahovat všechny citace na které je v textu práce odkazováno.  

\section{Překlad}
Pro kompilaci této ukázkové práce úplně od počátku\footnote{Anglicky build from scratch} je nutné provést několik spuštění pdf\LaTeX{}u a programu Biber v následujícím pořadí:
\begin{verbatim}
pdflatex <main file name>
biber <main file name>
pdflatex <main file name>
pdflatex <main file name>
pdflatex <main file name>
\end{verbatim}
\endinput