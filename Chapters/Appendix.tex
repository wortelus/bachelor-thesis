\chapter{Adresářová struktura a spouštění projektu}
Toto je kódová báze pro bakalářskou práci na téma ``Lokalizace klíčových bodů pomocí neuronových sítí''.

\section{Adresářová struktura}
\begin{itemize}
    \item \texttt{data/} -- složka obsahující vstupní data (prvních 100 snímků z datasetu \texttt{default/test.csv})
    \item \texttt{logs/} -- složka obsahující natrénované modely
    \item \texttt{metrics/} -- složka obsahující metriky
    \item \texttt{samples/} -- složka obsahující ukázková data
    \item \texttt{src/} -- zdrojové kódy
    \item \texttt{src/config/} -- konfigurační soubory
    \item \texttt{src/models/} -- definice modelů
    \item \texttt{src/plots} -- skripty pro vykreslování grafů
    \item \texttt{src/processing/} -- skripty pro zpracování dat
    \item \texttt{src/scripts/} -- pomocné skripty
    \item \texttt{src/utils/} -- pomocné funkce
    \item \texttt{src/start.py} -- hlavní skript pro spuštění
    \item \texttt{src/dinov2} -- implementace modelu DINOV2
\end{itemize}

\section{Příprava prostředí}
Před spouštením programu je potřeba nainstalovat závislosti pomocí příkazu:
\begin{verbatim}
pip install -r requirements.txt
\end{verbatim}
Úplně na začátek je třeba ověřit cesty v \texttt{src/config/paths.py}.

Model k trénování a ostatní parametry je třeba specifikovat a ověřit v \texttt{src/config}

\section{Spouštění programu}
Program se spouští pomocí skriptu \texttt{start.py}. Skript má:
\begin{itemize}
    \item první jednoslovný parametr pro výběr akce
    \item povinný parametr \texttt{-o} pro výběr pracovní/výstupní složky. Tento parametr navazuje také i na cesty definované v \texttt{src/config/paths.py}
\end{itemize}
\begin{verbatim}
python3 src/start.py inference -o .
\end{verbatim}

Během inference se vás skript zeptá na třídu a ID klíčového bodu.

Vykreslování inference je interaktivní, \texttt{plt.show()} očekává cíl, např. pomocí IDE.

\subsection{Možné akce}
\begin{itemize}
    \item \texttt{fit} -- natrénování modelu
    \item \texttt{inference} -- inference modelu
    \item \texttt{real} -- inference modelu na reálných datech
    \item \texttt{false\_detections} -- metrika pro falešné detekce
    \item \texttt{precision} -- metrika pro přesnost
    \item \texttt{single} -- tvorba datasetu pro jeden klíčový bod
    \item \texttt{partial} -- tvorba datasetu pro vybraný set klíčových bodů
    \item \texttt{complete} -- tvorba datasetu pro všechny klíčové body
    \item \texttt{convert} -- konverze checkpointu do modelu
    \item \texttt{samples} -- tvorba ukázkových dat
\end{itemize}

\endinput