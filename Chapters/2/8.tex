\section{Transformer STN}
\label{sec:Chapter28}

\begin{figure}[ht]
\centering
\includegraphics[width=0.8\textwidth,keepaspectratio]{Figures/stn/stn_module.pdf}
\caption[Zjednodušený pohled na STN modul]
{Zjednodušený pohled na STN modul, kde $U$ je vstupní obraz a $V$ je výstupní obraz. Lokalizační síť provádí regresi transformačních parametrů $\theta$, která je pak využita pro generaci vzorkového pole. Příkladná vizualizace vzorkového pole, a jeho použití ve vzorkovníku, je znázorněna na obrázku \ref{fig:stn_grid}. Převzato z \cite{stn} a upraveno. }
\label{fig:stn_overview}
\end{figure}

Architektura STN je jedním z možných přístupů jak představit spaciální invarianci do klasických konvolučních neuronových sítí představená v literatuře \cite{stn} roku 2015. Klasické konvoluční neuronové sítě pomocích svých vrstev typu max pooling \footnote{v síti U-Net nacházející se v enkodéru zachycující konktextuální informace} tvoří jistou spaciální invarianci proti různým změnám perspektivy, avšak pouze v limitovaném měřítku. Výrazné transformace vstupního obrazu mívá detrimentální účinek na výslednou robustnost sítě.

Pro adresování této limitace v oboru prostorové invariance vznikla neuronová síť s prostorovým transformerem - \textbf{STN (ang. Spatial Transformer Network)}. Z původního hlediska se často nejedná o hotové řešení, avšak o diferenciovatelný modul schopný být přidán do již existující architektury konvolučních neuronových sítí pro zvýšení odolnosti proti těmto prostorovým změnám a transformacím. Oproti použití klasických vrstev typu max pooling, tato síť je schopna se naučit a aktivně transformovat vstupní obraz do tohoto modulu. Modul také díky své diferenciovatelnosti může být triviálně natrénován pomocí klasické zpětné propagace. Modul STN sestává z několika částí (viděných na obrázku \ref{fig:stn_overview}):

\subsection{Lokalizační síť}

První částí je \textbf{lokalizační síť} (ang. localization net). Právě tato část modulu je jako jediná diferenciovatelná a trénovatelná. Vstupem do lokalizační sítě je snímek, který prane přímo ze vstupu modulu STN. Snímek může být více-kanálový, výstup lokalizační sítě je singulární a společný pro všechny kanály vstupního snímku. Architektura této části není striktně dána, může být postavena jak na CNN či FCN, s předpokladem že finální část sítě bude uzpůsobena pro regresi parametrů afinní transformace $\theta$ reprezentovanou v matici ${\displaystyle \mathrm {A} }_\theta$.

Matice ${\displaystyle \mathrm {A} }_\theta$ může nabírat několika forem. Prvním klasickým přístupem může být následující matice $2\times3$:
\begin{equation}
    {\displaystyle \mathrm {A} }_\theta = 
    \begin{bmatrix}
        \theta_{11} & \theta_{12} & \theta_{13} \\
        \theta_{21} & \theta_{22} & \theta_{23}
    \end{bmatrix}
    \label{eq:stn_6_theta}
\end{equation}
pro translaci, rotaci, škálování a zkosení podél os. Matice ${\displaystyle \mathrm {A} }_\theta$ avšak může nabírat i více podob, jako např. 12-člennou matici $4\times3$ pro 3D afinní transformace nebo následující poupravenou afinní 2D matici pro prostorovou pozornost, umožňující translaci, izotropní škálování a ořezávání:

\begin{equation}
    {\displaystyle \mathrm {A} }_\theta = 
    \begin{bmatrix}
        s & 0 & t_x \\
        0 & s & t_y
    \end{bmatrix}.
    \label{eq:stn_3_theta}
\end{equation}

Ořezávání je dosaženo s pomocí levé $2\times2$ strany matice ${\displaystyle \mathrm {A} }_\theta$ zapsanou jako ${\displaystyle \mathrm {A'} }_\theta$. Pokud platí, že $determinant({\displaystyle \mathrm {A'} }_\theta) < 1.0$, pak je výsledná plocha transformace menší než plocha originálního vstupu a vzorkové pole se soustředí pouze na jistou část snímku, což vede k lokálnější prostorové pozornosti.

\begin{figure}[h]
\centering
\includegraphics[width=0.3\textwidth,keepaspectratio]{Figures/stn/stn_a.png}
\includegraphics[width=0.3\textwidth,keepaspectratio]{Figures/stn/stn_b.png}
\caption[Vizualizace generování vzorkového pole v modulu STN]
{Vizualizace generování vzorkového pole v modulu STN a jeho aplikace na transformování vstupního snímku ve vzorkovníku, kde $T_{\theta}(G)$ je 2D skalární pole s transformovanými souřadnicemi pravidelné mřížky (ang. regular grid) $G$ zdrojového pole. $T_{I}(G)$ je jednotková transformace. Převzato z \cite{stn}. }
\label{fig:stn_grid}
\end{figure}

\subsection{Generátor vzorkového pole}

Výstup lokalizační sítě pak následuje do další části modulu STN - \textbf{generátoru vzorkového pole} (v originální ang. podobě grid generator). Generátor vzorkového pole generuje vzorkové pole, které aplikuje parametry afinní transformace ${\displaystyle \mathrm {A} }_\theta$. K tomu se využívá způsob inverzního mapování, což je klasickým přístupem mapování výstupního obrazu na vstupní obraz.

Inverzní mapování je často použito oproti přímému mapování, které nefunguje dobře díky přesahům a mezerám, které zanechá při mapování ve výsledném snímku. Inverzní mapování se obecně jedná o standardní postup v praxi při podobných úlohách počítačového vidění \cite{stn_medium_1}. Inverzní mapování v kontextu STN ve 2-dimenzionálním prostoru může být (identicky jako v originální literatuře \cite{stn}) definováno takto:

\begin{equation}
\begin{pmatrix}
x_i^s \\
y_i^s
\end{pmatrix}
= T_{\theta}(G_i) = {\displaystyle \mathrm {A} }_\theta
\begin{pmatrix}
x_i^t \\
y_i^t \\
1
\end{pmatrix}
= 
\begin{bmatrix}
\theta_{11} & \theta_{12} & \theta_{13} \\
\theta_{21} & \theta_{22} & \theta_{23}
\end{bmatrix}
\begin{pmatrix}
x_i^t \\
y_i^t \\
1
\end{pmatrix},
\label{eq:stn_inverse}
\end{equation}
kde $(x_i^s, y_i^s)$ je daná pozice na původním obrazu kterou počítáme pomocí funkce inverzního mapování na základě vzorkového pole $T_{\theta}(G_i)$. To lze vyjádřit jako násobení matice afinní transformace ${\displaystyle \mathrm {A} }_\theta$ a cílové pozici v homogenní formě $(x_i^t, y_i^t)$. Výsledkem použití inverzního mapování (ve funkci \ref{eq:stn_inverse}) jak bylo řečeno je vzorkového pole, reprezentováno jako 2D vektorové pole či dvě 2D skalární pole pro osy $(x, y)$.

\subsection{Vzorkovník}

Pro provedení prostorové transformace na originální snímek modulu STN zde máme poslední část známou jako \textbf{vzorkovník} (v originální ang. podobě sampler). Vzorkovník použije společně s vstupním snímkem a vzorkovým polem finální snímek modulu STN s aplikovanými prostorovými změnami. Vzorkovník využije vzorkové pole, a s pomocí bilineární interpolace převede pixely z originálního snímku na transformovaný snímek finální, podobně jak je znázorněno na \ref{fig:stn_grid}. Je dobré podotknout, že velikost vzorkového pole přímo ovlivňuje velikost výstupního snímku. Pokud má snímek více kanálů, je transformace aplikována identicky na všechny kanály vstupního snímku \cite{stn_medium_3}.

Převodová vzorkovací operace může být zjednodušeně pro mapování hodnot přímo na základě nejbližší lokace bez použití interpolace znázorněna takto:
\begin{equation}
    V_i^c = \sum_{n=1}^{H} \sum_{m=1}^{W} U_{nm}^c \delta(\lfloor x_s^i + 0.5 \rfloor - m) \delta(\lfloor y_s^i + 0.5 \rfloor - n),
\label{eq:stn_sampler_int}
\end{equation}
kde se zdrojové pozice $(x_i^s, y_i^s)$ mapují na výsledné pozice $(x_i^t, y_i^t)$ pro $V_i^c$, $U_{nm}^c$ je hodnota pixelu ve vstupním obrazu na dané pozici $(n, m)$ v kanálu $c$, $\delta$ je Kroneckerovo delta a pomocí posunu o hodnotě 0.5 se pozice zaokrouhlí na nejbližší celé číslo. Pro verzi s použitím bilineární interpolace se převod na základě předchozí funkce \ref{eq:stn_sampler_int} dá formulovat následovně:
\begin{equation}
    V_i^c = \sum_{n=1}^{H} \sum_{m=1}^{W} U_{nm}^c \max(0, 1 - |x_s^i - m|) \max(0, 1 - |y_s^i - n|).
\label{eq:stn_sampler_bi}
\end{equation}

\subsection{Zpětná propagace STN}

Funkce \ref{eq:stn_sampler_bi} pro vzorkovaní pozic ze zdroje na cíl je diferenciovatelná pro všechny hodnoty na výsledné mapě $V$, a je možno použít i jiné kernelové funkce\footnote{Označení pro funkci vzorkovníku \cite{stn}.} pro převod těchto pozic a hodnot, za předpokladu zachování diferenciovatelnosti pro účely zpětné propagace. Pro funkce pro výslednou mapu $V_i^c$ je možno odvodit následné parciální derivace vzhledem k zdrojovému obrazu $\frac{\delta V_i^c}{\delta U_{nm}^c}$, a také i zdrojovým pozicím $\frac{\delta V_i^c}{\delta x_i^s}$, resp. $\frac{\delta V_i^c}{\delta y_i^s}$, umožňující propagaci gradientů jak do zdrojového obrazu $U_{nm}^c$, tak pozic vzorkového pole $(x_i^s, y_i^s)$. Zpětná propagace díky tomu dosahuje i do výsledků lokalizační sítě, parametrů $\theta$, jelikož parciální derivace $\frac{\delta x_i^s}{\delta \theta}$ a $\frac{\delta y_i^s}{\delta \theta}$ jsou schopny být odvozeny a zderivovány z funkce \ref{eq:stn_inverse}. Díky této vlastnosti může být modul STN vložen do existujících sítí a být jednoduše natrénován v rámci sítě jako celku (tzv. end-to-end\footnote{Proces trénování celého modelu jako jednoho celku od vstupu po výstup.}) \cite{stn}. 
\endinput