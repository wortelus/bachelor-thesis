\chapter{Metody lokalizace klíčových bodů}
\label{sec:Chapter2}
V této kapitole provedeme rešerši na různé metody počítačového vidění pro lokalizaci klíčových bodů. Lokalizaci klíčových bodů na 2D obrázcích můžeme provést klasickými algoritmickými přístupy, jako je například SIFT, který algoritmicky generuje klíčové body na základě distinktivních vlastností obrázků. 

Táto práce se však více bude zaměřovat na využití konvolučních neuronových sítí (CNN) a architektur z řad U-Net pro specializovaně natrénované modely lokalizující klíčové body v řadách milisekund. Následně si také přiblížíme i modul STN, který je možno použít pro řešení problému prostorové invariance v přístupech pomocí konvolučních neuronových sítích. V této kapitole jsou zmíněny i modely state-of-art, jako je například YOLOv8 nebo DINOv2, které jsou více obecné a univerzálnější přístupy pro úlohy počítačového vidění v posledních letech.
\endinput