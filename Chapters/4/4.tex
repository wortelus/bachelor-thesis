\section{Řešení pomocí sítě U-Net++}
\label{sec:Chapter44}
Síť U-Net++ se odvíjí od architektury U-Net v předchozí kapitole \ref{sec:Chapter43}, rozšiřuje její možnosti a sémanticky propojuje části enkodéru a dekodéru svými skokovými bloky \cite{unetpp}. Jako v předchozí kapitole, U-Net++ obsahuje 4 bloky enkodéru a dekodéru. Enkodér taktéž začíná na počtu 32 filtrů až po počet 256 filtrů na posledním bloku. Krk má 512 filtrů. Dekodér je symetricky konstruován k enkodéru. Obecné návrhy popsány v kapitole \ref{sec:Chapter42} jsou obdobně použity i pro hluboké skokové bloky přidány v rámci sítě U-Net++ \cite{unetpp}.

Skokové bloky jsou přidány mezi vrstvy enkodéru a dekodéru v rámci pokusu vylepšení výsledků původní sítě U-Net. Premise a detaily vylepšení této sítě jsou detailně popsány v kapitole \ref{sec:Chapter23}. Přesné vyobrazení implementované sítě je graficky reprezentováno v části (b) obrázku \ref{fig:unetpp}. Inspirace pro implementační detaily této sítě plyne také z open-source implementace citované v originální literatuře \cite{unetpp_github}.

Síť je trénována v režimu hluboké supervize na základě výstupu bloků $X^{0,1}$, $X^{0,2}$, $X^{0,3}$, $X^{0,4}$ (jejich ztrátové hodnoty jsou sečteny a tato hodnota je minimalizována optimizérem). Samozřejmě ztrátová funkce nepracuje s ReLU výstupy, ale pro každý blok účastnící se hluboké supervize byla přidána $1\times1$ finální konvoluční vrstva. Při trénování modelu bez hluboké supervize pomocí standardního přístupu zpětné propagace na finální vrstvě se ztrátové funkce účastní pouze finální blok dekodéru $X^{0, 4}$.

Pro U-Net++ se v režimu hluboké supervize očekává průměrování na výsledných výstupech zúčastněných bloků. Tato možnost bude vyzkoušena v implementační a evaluační části práce a porovnána s výsledky výstupu jednotného bloku $X^{0, 4}$ také v rámci hluboké supervize.

Síť U-Net++ obsahuje ve verzi s hlubokou supervizí 9 164 748 parametrů a 9 163 659 v režimu bez hluboké supervize, což je zhruba 17\% nárůst velikosti sítě oproti síti U-Net. Velikost v obou případech dosahuje velikosti zhruba 35 MB.
\endinput