\section{Řešení pomocí sítě U-Net STN}
\label{sec:Chapter45}
U-Net STN není přesným názvem pro ustálenou architekturu neuronové sítě, avšak název zvolen v rámci této práce pro experiment o zařazení modulu STN (popsaného v kapitole \ref{sec:Chapter28}) do sítě U-Net (popsanou v kapitole \ref{sec:Chapter22}). Pomocí přirozeného úsudku jsem zvolil modul STN vložit jako první část sítě, předcházející enkodér a následné vrstvy. Premisí zařazení tohoto modulu, které by mohly vylepšit výsledky původní architektury U-Net, může být hned několik:
\begin{enumerate}
    \item \textbf{Lepší prostorová invariance} -- Jelikož klasické přístupy pomocí konvolučních neuronových sítí (CNN) relativně postrádají invarianci vůči prostorovým změnám \cite{stn}, modul STN se může naučit převést vstupní snímek do více generalizované formy, jinak řečeno lépe \enquote{chápatelné} a zpracovatelné pro následující vrstvy sítě navazující na výstup modulu STN. Toto také může teoreticky vést i ke zlehčení lokalizační a klasifikační úlohy pro následnou síť U-Net díky přenosu této zodpovědnosti ze sítě U-Net do modulu STN. Tímto se tyto 2 obecné části mohou k sobě stát komplementárními, což zlepšuje obecnou robustnost této sítě jako celku.
    \item \textbf{Představení mechanizmu pozornosti} -- Modul STN se může naučit během tréninku lépe identifikovat důležité části snímku a předpovědět, v jaké oblasti na snímku bude pravděpodobně probíhat lokalizační úloha. Díky tomu se může modul několika způsoby pokusit o představení pozornosti na tuto oblast, a to jak například pomocí přiblížení na tuto oblast (pomocí škálovacích parametrů transformační matice $\theta$), tak i pomocí přesunutí této oblasti do středu výstupní mapy.
    \item \textbf{Ko-lokalizátor} -- Funkce modulu STN jako ko-lokalizátor souvisí s mechanizmem pozornosti a jedná se tedy pouze o přímou implikaci této schopnosti na možnost již už také hrubě rozeznat klíčové body. I když trénovatelná část modulu STN není nijak propojená s následnou sítí U-Net a nemohou sdílet informace v tradičním smyslu propojených vrstev neuronových sítí, modul STN může již jednoduché lokalizační úlohy hrubě lokalizovat a následně je například převést do středu snímku.
\end{enumerate}

Pro pokus o zařazení modulu STN může být zvolen přístup kompletní afinní 2D transformace pomocí 6 parametrů $\theta$ (matice \ref{eq:stn_6_theta}), nebo pomocí pouze 3 parametrů $\theta$ (matice \ref{eq:stn_3_theta}). Obraz je pomocí vzorkovníku na daných parametrech transformován a předán dále do sítě U-Net, tedy do prvního bloku enkodéru. Síť U-Net je v tomto přístupu stavěna identicky jako v předchozím přístupu v kapitole \ref{sec:Chapter43}.

\subsection{Návrh lokalizační sítě STN}

\textbf{Lokalizační síť} modulu STN je v případě této práce navržena jako síť typu CNN s navazující regresní sítí plně propojených vrstev (ang. dense layer). V následující tabulce \ref{fig:stn_loc_net} je popsána zvolená architektura. $m$ je referenční hodnota pro počet jednotek a filtrů, stanovena na hodnotu $128$ a $n_{\theta}$ zase určuje počet výstupních parametrů $\theta$, v rámci praktické části této práce $3$ a $6$:

\begin{table}[H]
\centering
\begin{tabular}{@{}llrr@{}}
\toprule
Typ vrstvy & Aktivační funkce & Počet filtrů / jednotek neuronů & Velikost filtru \\ \midrule
Conv2D & ReLU & $m$ & $3 \times 3$ \\
MaxPooling2D & - & - & $2 \times 2$ \\
Conv2D & ReLU & $2\times m$ & $3 \times 3$ \\
MaxPooling2D & - & - & $2 \times 2$ \\
Conv2D & ReLU & $4\times m$ & $3 \times 3$ \\
MaxPooling2D & - & - & $2 \times 2$ \\
GlobalAverageMaxPooling & - & - & - \\
\bottomrule
Dense & ReLU & $4\times m$ & - \\
Dense & ReLU & $2\times m$ & - \\
Dense & ReLU & $n_{\theta}$ & - \\
\bottomrule
\end{tabular}
\caption[Návrh lokalizační sítě modulu STN] { Návrh lokalizační sítě modulu STN }
\label{fig:stn_loc_net}
\end{table}

\subsection{Návrh vzorkového pole}

V modulu STN je v druhé částí generátor vzorkového pole. Generátor vzorkového pole má zodpovědnost generovat vzorkovou mapu $T_{\theta}(G)$ z jednotkové vzorkové mapy $T_{I}(G) = G$ (viz obrázek \ref{fig:stn_grid}), která je v normalizovaném rozsahu $\langle-1, 1\rangle$ pro určení středu souřadného systému doprostřed snímku. Vzorková mapa $T_{\theta}(G)$ se generuje pomocí matice $\displaystyle \mathrm {A}_\theta$. 

Výsledné vzorkové pole $T_{\theta}(G)$ následně vstupuje do vzorkovníku, který na vstupní snímek aplikuje pomocí bilineární interpolace finální transformaci generující výstupní snímek modulu STN.

\subsection{Způsob tréninku}

Síť U-Net STN může být natrénovaná několika způsoby. V datasetu jsou dostupné syntetické snímky s již transformovaným klíčovým bodem do normalizované podoby, včetně komplementární informace o rotaci a škálování použité k transformaci pravých snímků do syntetických. Modul STN můžeme trénovat separátně, nebo také i v tzv. režimu end-to-end v rámci celé sítě U-Net STN.

Pro U-Net STN byl zvolen přístup v režimu \textbf{tzv. end-to-end a bez přímé supervize} čili jsme se rozhodli nepoužít dodatečné informace datasetu a parametry $\theta$ přímě neaplikovat ve výpočtu ztrátové funkce. Modul STN se ve fázi tréninku sítě tedy pokusí o nalezení nejlepšího způsobu transformací pro finální lokalizační úlohu.

\endinput