\section{Řešení pomocí sítě U-Net}
\label{sec:Chapter43}
Jako náš první základní model volíme síť standardní U-Net architektury navazující na původní architekturu představenou v \cite{unet}. Byl zvolen přístup 4 bloků jak enkodéru, tak symetricky i dekodéru. Architektura může být znázorněna pomocí následující tabulky \ref{fig:model_architecture}:

\begin{table}[ht]
\centering
\begin{tabular}{@{}lrr@{}}
\toprule
Typ vrstvy & Počet filtrů & Velikost filtru \\ \midrule
$2\times$ Conv2D+BN+ReLU & 32 & $3 \times 3$ \\
MaxPooling2D & - & $2 \times 2$ \\
$2\times$ Conv2D+BN+ReLU & 64 & $3 \times 3$ \\
MaxPooling2D & - & $2 \times 2$ \\
$2\times$ Conv2D+BN+ReLU & 128 & $3 \times 3$ \\
MaxPooling2D & - & $2 \times 2$ \\
$2\times$ Conv2D+BN+ReLU & 256 & $3 \times 3$ \\
MaxPooling2D & - & $2 \times 2$ \\
$2\times$ Conv2D+BN+ReLU & 512 & $3 \times 3$ \\
UpSampling2D & - & $2 \times 2$ \\
$2\times$ Conv2D+BN+ReLU & 256 & $3 \times 3$ \\
UpSampling2D & - & $2 \times 2$ \\
$2\times$ Conv2D+BN+ReLU & 128 & $3 \times 3$ \\
UpSampling2D & - & $2 \times 2$ \\
$2\times$ Conv2D+BN+ReLU & 64 & $3 \times 3$ \\
UpSampling2D & - & $2 \times 2$ \\
$2\times$ Conv2D+BN+ReLU & 32 & $3 \times 3$ \\
% Final output layer
Conv2D+Sigmoid & 11 & $1 \times 1$ \\
\bottomrule
\end{tabular}
\caption[Konvoluční a pooling vrstvy implementace sítě U-Net]{Konvoluční a pooling vrstvy implementace sítě U-Net, kde Conv2D je konvoluční operace a BN je zkratkou pro dávkovou normalizaci (ang. batch normalization). }
\label{fig:model_architecture}
\end{table}

Ve finále tato síť bez parametrů optimizéru či uložených gradientů dosahuje velikosti 7 852 875 parametrů (okolo 30 MB).


\endinput