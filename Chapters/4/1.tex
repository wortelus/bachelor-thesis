\section{Augmentace dat}
\label{sec:Chapter41}
Pro účely trénování všech implementovaných neuronových sítí v této práci bylo využito augmentace trénovacích dat. Augmentace těchto snímků slouží pro představení náhodně upravených grafických odchylek během zpětné propagace sítě. Slouží pro syntetické zvětšení rozsahu datasetu a představení jistých neperfektních elementů. Tento častý přístup pak slouží k více robustní schopnosti následně vykonávat svůj úkol na reálných snímcích. Identický postup augmentace dat je použit i během následných evaluacích modelů. Mezi tyto augmentace patří:
\begin{enumerate}
  \item \textbf{Barevnost} (ang. saturation) - úprava na základě odchylky max. 10 \% oproti původní hodnotě
  \item \textbf{Světelnost} (ang. brightness) - úprava na základě odchylky max. 10 \% oproti původní hodnotě
  \item \textbf{Barevná škála} (ang. hue) - úprava RGB škály na základě odchylky max. 5 \% oproti původní hodnotě
  \item \textbf{Gaussovo rozostření} - aplikováno pomocí následujícího rozdělení
  \begin{enumerate}
      \item 10\% šance na aplikování Gaussova rozostření pomocí filtru $3\times3, \sigma=$1,0
      \item 10\% šance na aplikování Gaussova rozostření pomocí filtru $5\times5, \sigma=$1,0
  \end{enumerate}
  \item \textbf{Náhodný šum} - úprava pixelů o maximálně 3 \% (0.03) hodnoty $\sigma$ směrodatné odchylky.
\end{enumerate}
\endinput